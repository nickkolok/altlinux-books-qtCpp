\setcounter{tocdepth}{10}
\renewcommand\contentsname{Оглавление}
\tableofcontents
\section{}
\clearpage\section[Приложение. Общие сведения о библиотеке MathGL]{Приложение. Общие сведения о библиотеке MathGL}
При решении различных задач, возникает необходимости графического отображения данных (графики, диаграммы, поверхности).
Одним из универсальных способов построения различных графиков является кросс-платформенная библиотека MathGL.

\index{Библиотека!Mathgl}\textstyleEmphasis{Mathgl} – свободная кросс платформенная библиотека для построения 2-х и 3-х
мерных графиков функции. Её использование позволит построить график с помощью нескольких операторов. Синтаксис функций,
используемых в MathGL, подобен синтаксису MathLab, Octave, Scilab, GnuPlot. Официальный сайт – 
http://mathgl.sourceforge.net/doc\_ru/Website.html\#Website, на странице загрузки
http://mathgl.sourceforge.net/doc\_ru/Download.html\#Download можно скачать последнюю версию программы для различных
ОС, англоязычную документацию в формате pdf. Группа в Google – https://groups.google.com/forum/\#!forum/mathgl.
Русскоязычная страница с описанием – http://mathgl.sourceforge.net/doc\_ru/index.html\#SEC\_Contents, англоязычная –
http://mathgl.sourceforge.net/doc\_en/index.html\#SEC\_Contents. Кроме С(С++) поддерживаются Fortran, Python, Octave,
скриптовый язык MGL.

Рассмотрим особенности установки и примеры использования библиотеки для построения графиков.

\subsection{Установка MathGL в Linux.}
Библиотека входит в репозитории большинства современных дистрибутивов Linux. Её можно установить из репозитория
стандартным для вашего дистрибутива, однако в репозитории зачастую находится не самая новая версия. Для установки самой
новой версии необходимо:

\begin{enumerate}
\item Скачать исходники последней версии с официального сайта.
\item Распаковать.
\item Последовательно выполнить команды\footnote{Перед выполнением команды cmake, возможно, придётся доставлять
необходимые пакеты. При работе в debian (6, 7), ubuntu (12.04, 12.10, 13.04, 13.10) авторам пришлось доставить  пакеты
cmake, zlib1g-dev, libpng12-dev, libqt4-opengl-dev, libqtwebkit-dev. Кроме того, должен быть устанвлен компилятор g++.}
\end{enumerate}
\begin{itemize}
\item \begin{itemize}
\item cmake -Denable-qt=ON .
\item cmake .
\item make
\item sudo make install
\end{itemize}
\end{itemize}
\begin{enumerate}
\item Скопировать файлы libmgl-qt.so.7.1.0 и libmgl.so.7.1.0\footnote{Версии библиотек libmgl указаны применительно к
mathgl 2.2, в вашем конкретном случае могут быть другие библиотеки.} из каталога /usr/local/lib в каталог /lib (нужны
права администратора(суперпользователя).
\end{enumerate}
После этого для компиляции программы с использованием библиотеки MathGL необходимо использовать ключи -L/usr/local/lib
-L/usr/lib -lmgl -lmgl-qt, например для компиляции файла с именем 1.cpp можно использовать команду

{\upshape
g++ -o 1  1.cpp -L/usr/local/lib -L/usr/lib -lmgl -lmgl-qt}

Рассмотрим возможности библиотеки на конкретных примерах.

ЗАДАЧА {\refstepcounter{qwerty}\theqwerty\label{seq:ref0}}. Построить график функции  $f(x)=\sin (x)+\frac{1}{3}\cdot
\sin (3\cdot x)+\frac{1}{5}\cdot \sin (5\cdot x)$ .

Следующий программный код позволит построить линию графика (рис. 1) на интервале [-1;1].

\#include {\textless}mgl2/qt.h{\textgreater}

int sample(mglGraph *gr)

\{

\ \ //Заголоок графика

\ \ gr-{\textgreater}Title(null{<<}График функции y=f(x)null{<<});

\ \ //График заданной функции.

{\upshape
\ \ //Линия графика изображена красным цветом {}- null{<<}rnull{<<}.}

\ \ gr-{\textgreater}FPlot(null{<<}sin(x)+1/3*sin(3*x)+1/5*sin(5*x)null{<<},null{<<}rnull{<<});

return 0;

\}

int main(int argc,char **argv)

\{

setlocale(LC\_CTYPE, null{<<}ru\_RU.utf8null{<<});

mglQT gr(sample,null{<<}Plotnull{<<});

return gr.Run();

\}

Далее представлен текст программы, с помощью которого можно усовершенствовать график, показанный на рис.
\ref{seq:ref0a}. На рис. \ref{seq:ref1} видно, что был увеличен диапазон построения графика, добавлены оси координат,
подписи под ними и линии сетки.

\begin{minipage}{10.961cm}
{\itshape
Рис. {\refstepcounter{qwertya}\theqwertya\label{seq:ref0a}}: График функции к задаче \ref{seq:ref0}}
\includegraphics[scale=0.33]{mgl-img001}\end{minipage}

{\upshape
\#include {\textless}mgl2/qt.h{\textgreater}}

{\upshape
int sample(mglGraph *gr)}

{\upshape
\{}

{\upshape
\ \ //Заголоок графика}

{\upshape
\ \ gr-{\textgreater}Title(null{<<}График функции y=f(x)null{<<});}

{\upshape
\ \ //Установка центра координатных осей}

{\upshape
\ \ gr-{\textgreater}SetOrigin(0,0);}

{\upshape
\ \ //Границы по оси абсцисс от -10 до 10,}

\ \ //по оси ординат от -1 до 1.

{\upshape
\ \ gr-{\textgreater}SetRanges(-10,10,-1,1);}

{\upshape
\ \ //Вывод значений возле осей}

{\upshape
\ \ gr-{\textgreater}Axis();}

{\upshape
\ \ //Линии сетки}

{\upshape
\ \ gr-{\textgreater}Grid();}

\ \ //График заданной функции.

{\upshape
\ \ gr-{\textgreater}FPlot(null{<<}sin(x)+1/3*sin(3*x)+1/5*sin(5*x)null{<<}, null{<<}rnull{<<});}

{\upshape
\ \ return 0;}

{\upshape
\}}

{\upshape
int main(int argc,char **argv)}

{\upshape
\{\ \ //Поддержка кириллицы в С++}

{\upshape
\ \ setlocale(LC\_CTYPE, null{<<}ru\_RU.utf8null{<<});}

{\upshape
\ \ //Вывод графика на экран в окно с именем Plot}

{\upshape
\ \ mglQT gr(sample,null{<<}Plotnull{<<});}

{\upshape
\ \ return gr.Run();}

{\upshape
\}}

{\centering \begin{minipage}{14.081cm}
{\itshape
Рис. {\refstepcounter{qwertya}\theqwertya\label{seq:ref1}}: Форматирование линии и области построения графика к задаче
\ref{seq:ref0}}
\includegraphics[scale=0.33]{mgl-img002}\end{minipage}\par}

ЗАДАЧА {\refstepcounter{qwerty}\theqwerty\label{seq:ref1a}}. Построить графики функций  $f(x)=\sin (2\cdot x)$ и 
$y(x)=\frac{4\cdot \cos (x)}{3}$  в одной графической области.

Далее приведен текст программы реализующий решение поставленной задачи. Результаты работы программы показаны на рис.
\ref{seq:ref2}.

\#include {\textless}mgl2/qt.h{\textgreater}

\#include {\textless}math.h{\textgreater}

int sample(mglGraph *gr)

{\upshape
\{\ \ //Заголовок графика}

\ \ gr-{\textgreater}Title(null{<<}Графики функции y=f(x)null{<<});

{\upshape
\ \ //Границы по осям}

\ \ gr-{\textgreater}SetRanges(-15,15,-2,2);

{\upshape
\ \ //Вывод значений возле осей}

\ \ gr-{\textgreater}Axis();

{\upshape
\ \ //Линии сетки}

\ \ gr-{\textgreater}Grid();

\ \ //График функции f(x), красная (r) сплошная линия.

\ \ gr-{\textgreater}FPlot(null{<<}sin(2*x)null{<<}, null{<<}rnull{<<});

\ \ //Добавление легенды

\ \ gr-{\textgreater}AddLegend(null{<<}sin(2*x)null{<<},null{<<}rnull{<<});

{\upshape
\ \ //График функции y(x), чёрная (k) линия и точки (.).}

\ \ gr-{\textgreater}FPlot(null{<<}4*cos(x)/3null{<<}, null{<<}k.null{<<});

\ \ //Добавление легенды

\ \ gr-{\textgreater}AddLegend(null{<<}4*cos(x)/3null{<<},null{<<}k.null{<<});

{\upshape
\ \ //Вывод легенды на экран в правом верхнем углу}

\ \ gr-{\textgreater}Legend(3);

{\upshape
\ \ //Вывод подписи по оси абсцисс}

\ \ gr-{\textgreater}Label('x',null{<<}OXnull{<<},0);

{\upshape
\ \ //Вывод подписи по оси ординат}

\ \ gr-{\textgreater}Label('y',null{<<}OYnull{<<});

\ \ return 0;

\}

int main(int argc,char **argv)

\{

\ \ setlocale(LC\_CTYPE, null{<<}ru\_RU.utf8null{<<});

\ \ mglQT gr(sample,null{<<}Plotnull{<<});

\ \ return gr.Run();

\}

{\centering \begin{minipage}{13.113cm}
{\itshape
Рис. {\refstepcounter{qwertya}\theqwertya\label{seq:ref2}}: Графики к задаче \ref{seq:ref1a}}
\includegraphics[scale=0.33]{mgl-img003}\end{minipage}\par}

ЗАДАЧА \stepcounter{qwerty}{\theqwerty}. Построить в одном графическом окне графики функций:

\begin{itemize}
\item  $f(x)=\sin (x)$ на интервале [-10;10], 
\item  $f(x)=\cos (x)$ на интервале [-6;6], 
\item  $f(x)=e^{\cos (x)}$ на интервале [-6;6],
\item  $f(x)=e^{\sin (x)}$ на интервале [-12;12].
\end{itemize}
Результаты вывести на экран и в файл.

Далее приведен программный код и результат его работы (рис. \ref{seq:ref3}).

{\upshape
\#include {\textless}mgl2/qt.h{\textgreater}}

{\upshape
\#include {\textless}mgl2/mgl.h{\textgreater}}

{\upshape
\#include {\textless}iostream{\textgreater}}

{\upshape
using namespace std;}

{\upshape
int sample(mglGraph *gr)}

{\upshape
\{}

{\upshape
//График функции sin(x) на интервале [-10;10]}

{\upshape
gr-{\textgreater}SubPlot(2,2,0); }

{\upshape
gr-{\textgreater}Title(null{<<}График функции sin(x)null{<<});}

{\upshape
gr-{\textgreater}SetOrigin(0,0);}

{\upshape
gr-{\textgreater}SetRanges(-10,10,-1,1);}

{\upshape
gr-{\textgreater}Axis(); }

{\upshape
gr-{\textgreater}Grid(); }

{\upshape
gr-{\textgreater}FPlot(null{<<}sin(x)null{<<},null{<<}k-.null{<<});}

{\upshape
//График функции cos(x) на интервале [-6;6]}

{\upshape
gr-{\textgreater}SubPlot(2,2,1); }

{\upshape
gr-{\textgreater}Title(null{<<}График функции cos(x)null{<<});}

{\upshape
gr-{\textgreater}SetOrigin(0,0);}

{\upshape
gr-{\textgreater}SetRanges(-6,6,-1,1);}

{\upshape
gr-{\textgreater}Axis(); }

{\upshape
gr-{\textgreater}Grid(); }

{\upshape
gr-{\textgreater}FPlot(null{<<}cos(x)null{<<},null{<<}k .null{<<});}

{\upshape
//График функции exp(cos(x)) на интервале [-6;6]}

{\upshape
gr-{\textgreater}SubPlot(2,2,2);}

{\upshape
gr-{\textgreater}Title(null{<<}График функции e\^{}\{cos(x)\}null{<<});}

{\upshape
gr-{\textgreater}SetOrigin(0,0);}

{\upshape
gr-{\textgreater}SetRanges(-6,6,0,3);}

{\upshape
gr-{\textgreater}Axis(); }

{\upshape
gr-{\textgreater}Grid(); }

{\upshape
gr-{\textgreater}FPlot(null{<<}exp(cos(x))null{<<},null{<<}r onull{<<});}

{\upshape
//График функции exp(sin(x)) на интервале [-12;12]}

{\upshape
gr-{\textgreater}SubPlot(2,2,3); }

{\upshape
gr-{\textgreater}Title(null{<<}График функции e\^{}\{sin(x)\}null{<<});}

{\upshape
gr-{\textgreater}SetOrigin(0,0);}

{\upshape
gr-{\textgreater}SetRanges(-15,15,0,3);}

{\upshape
gr-{\textgreater}Axis(); }

{\upshape
gr-{\textgreater}Grid(); }

{\upshape
gr-{\textgreater}FPlot(null{<<}exp(sin(x))null{<<},null{<<}r-onull{<<});}

{\upshape
return 0;}

{\upshape
\}}

{\upshape
int main(int argc,char **argv)}

{\upshape
\{}

{\upshape
\ \ //Вывод на экран или в файл}

{\upshape
\ \ int k;}

{\upshape
\ \ cout{\textless}{\textless}null{<<}Введите 1, если будете выводить на экран, 2 - если в
файл{\textbackslash}nk=null{<<};}

{\upshape
\ \ cin{\textgreater}{\textgreater}k;}

{\upshape
\ \ if(k==1)}

{\upshape
\ \ \{}

{\upshape
\ \ \ \ //Поддержка кириллицы в С++\ \ }

{\upshape
\ \ \ \ setlocale(LC\_CTYPE, null{<<}ru\_RU.utf8null{<<});}

{\upshape
\ \ \ \ //Вывод на экран}

{\upshape
\ \ \ \ mglQT gr(sample,null{<<}Plotsnull{<<});}

{\upshape
 \ \ \ \ return gr.Run();}

{\upshape
\ \ \}else}

{\upshape
\ \ \{}

{\upshape
\ \ \ \ mglGraph gr;}

{\upshape
\ \ \ \ gr.Alpha(true);}

{\upshape
\ \ \ \ gr.Light(true);}

{\upshape
\ \ \ \ setlocale(LC\_CTYPE, null{<<}ru\_RU.utf8null{<<});}

{\upshape
\ \ \ \ //Обращение к функции вывода}

{\upshape
\ \ \ \ sample(\&gr);}

{\upshape
\ \ \ \ //Запись изображения в файл}

{\upshape
\ \ \ \ gr.WriteEPS(null{<<}test.epsnull{<<});}

{\upshape
\ \ \ \ return 0;\}}

{\upshape
\}}

{\centering \begin{minipage}{14.864cm}
{\itshape
Рис. {\refstepcounter{qwertya}\theqwertya\label{seq:ref3}}: График функций к задаче 3}
\includegraphics[scale=0.33]{mgl-img004}\end{minipage}\par}

ЗАДАЧА {\refstepcounter{qwerty}\theqwerty\label{seq:ref3a}}. Построить график функций 
$f(x)=1-\frac{0.4}{x}+\frac{0.05}{x^{2}}$ .

Не трудно заметить, что функция \textstyleEmphasis{f}(\textstyleEmphasis{x}) не существует в точке ноль. Поэтому
построим её график на двух интервалах [-2;-0.1] и [0.1;2], исключив точку разрыва из диапазона построения. Текст
программы с подробными комментариями приведен далее. Решение задачи представлено на рис. \ref{seq:ref4}.

\#include {\textless}mgl2/qt.h{\textgreater}

\#include {\textless}iostream{\textgreater}

using namespace std;

int sample(mglGraph *gr)

\{\ \ mglData x1(191),x2(191), y1(191), y2(191) ;

\ \ int i; float h,a1,b1,a2,b2;

\ \ //График точечной разрывной функции

{\upshape
\ \ //Первый интервал}

\ \ a1=-2;b1=-0.1;

\ \ h=0.01;

\ \ for(i=0;i{\textless}191;i++)

\ \ \{

\ \ \ \ x1[i]=a1+i*h;

\ \ \ \ y1[i]=1-0.4/x1[i]+0.05/x1[i]/x1[i];

\ \ \}

\ \ //Второй интервал

\ \ a2=0.1;b2=2;

\ \ h=0.01;

\ \ for(i=0;i{\textless}191;i++)

\ \ \{

\ \ \ \ x2[i]=a2+i*h;

\ \ \ \ y2[i]=1-0.4/x2[i]+0.05/(x2[i]*x2[i]);

\ \ \}

\ \ //Границы по оси абсцисс и ординат

\ \ gr-{\textgreater}SetRanges(a1,b2,0,10);

{\upshape
\ \ //Оси координат}

\ \ gr-{\textgreater}Axis();

{\upshape
\ \ //Сетка}

\ \ gr-{\textgreater}Grid();

{\upshape
\ \ //График функции на первом интервале, чёрный (k) цвет.}

\ \ gr-{\textgreater}Plot(x1,y1,null{<<}knull{<<});

{\upshape
\ \ //График функции на втором интервале, чёрный (k) цвет.}

\ \ gr-{\textgreater}Plot(x2,y2,null{<<}knull{<<});

\ \ //Размер шрифта

{\upshape
\ \ gr-{\textgreater}SetFontSize(2);}

\ \ //Заголовок

\ \ gr-{\textgreater}Title(null{<<}График разрывной функцииnull{<<});

\ \ //Размер шрифта

\ \ gr-{\textgreater}SetFontSize(4);

\ \ //Подпись по оси абсцисс

\ \ gr-{\textgreater}Label('x',null{<<}OXnull{<<},0);

\ \ //Подпись по оси ординат

\ \  gr-{\textgreater}Label('y',null{<<}OYnull{<<});

return 0;

\}

int main(int argc,char **argv)

\{

\ \ setlocale(LC\_CTYPE, null{<<}ru\_RU.utf8null{<<});

\ \ mglQT gr(sample,null{<<}Plotnull{<<});

\ \ return gr.Run();

\}

\begin{minipage}{15.956cm}
{\itshape
Рис. {\refstepcounter{qwertya}\theqwertya\label{seq:ref4}}: График функции к задаче \ref{seq:ref3a}}
\includegraphics[scale=0.33]{mgl-img005}\end{minipage}

ЗАДАЧА {\refstepcounter{qwerty}\theqwerty\label{seq:ref4a}}. Построить график функции . $y(x)=\frac{1}{x^{2}-2\cdot
x-3}$ на интервале [-5;5]. Функция y(x) имеет разрыв в точках -1 и 3. Построим её график на трёх интервалах [-5;-1.1] и
[-0.9; 2.9], [3.1;5], исключив точки разрыва из диапазона построения. Текст программы с подробными комментариями
приведён далее. Решение задачи представлено на рис. \ref{seq:ref4}.

\#include {\textless}mgl2/qt.h{\textgreater}

\#include {\textless}iostream{\textgreater}

using namespace std;

int sample(mglGraph *gr)

\{\ \ mglData x1(391),x2(381), x3(191),y1(391), y2(381),y3(191) ;

\ \ int i; float h,a1,b1,a2,b2,a3,b3;

\ \ //График точечной разрывной функции

\ \ //Первый интервал

\ \ a1=-5;b1=-1.1;

\ \ h=0.01;

\ \ for(i=0;i{\textless}391;i++)

\ \ \{

\ \ \ \ x1[i]=a1+i*h;

\ \ \ \ y1[i]=1/(x1[i]*x1[i]-2*x1[i]-3);

\ \ \}

\ \ //Второй интервал

\ \ a2=-0.9;b2=2.9;

\ \ h=0.01;

\ \ for(i=0;i{\textless}381;i++)

\ \ \{

\ \ \ \ x2[i]=a2+i*h;

\ \ \ \ y2[i]=1/(x2[i]*x2[i]-2*x2[i]-3);

\ \ \}

\ \ //Третий интервал

\ \ a3=3.1;b3=5;

\ \ h=0.01;

\ \ for(i=0;i{\textless}191;i++)

\ \ \{

\ \ \ \ x3[i]=a3+i*h;

\ \ \ \ y3[i]=1/(x3[i]*x3[i]-2*x3[i]-3);

\ \ \}

\ \ //Границы по оси абсцисс и ординат

\ \ gr-{\textgreater}SetRanges(-6,6,-3,3);

\ \ //Оси координат

\ \ gr-{\textgreater}Axis();

\ \ //Сетка

\ \ gr-{\textgreater}Grid();

\ \ //График функции на первом интервале, чёрный (k) цвет.

\ \ gr-{\textgreater}Plot(x1,y1,null{<<}knull{<<});

\ \ //График функции на втором интервале, чёрный (k) цвет.

\ \ gr-{\textgreater}Plot(x2,y2,null{<<}knull{<<});

\ \ //График функции на третьем интервале, чёрный (k)

\ \ // цвет.

\ \ gr-{\textgreater}Plot(x3,y3,null{<<}knull{<<});

\ \ //Размер шрифта

\ \ gr-{\textgreater}SetFontSize(2);

\ \ //Заголовок

\ \ gr-{\textgreater}Title(null{<<}График функции c двумя разрывамиnull{<<});

\ \ //Размер шрифта

\ \ gr-{\textgreater}SetFontSize(4);

\ \ //Подпись по оси абсцисс

\ \ gr-{\textgreater}Label('x',null{<<}OXnull{<<},0);

\ \ //Подпись по оси ординат

\ \  gr-{\textgreater}Label('y',null{<<}OYnull{<<});

return 0;

\}

int main(int argc,char **argv)

\{

\ \ setlocale(LC\_CTYPE, null{<<}ru\_RU.utf8null{<<});

\ \ mglQT gr(sample,null{<<}Plotnull{<<});

\ \ return gr.Run();

\}

\begin{minipage}{15.956cm}
{\itshape
Рис. \stepcounter{qwertya}{\theqwertya} График функции к задаче \ref{seq:ref4a}}
\includegraphics[scale=0.33]{mgl-img006}\end{minipage}

ЗАДАЧА {\refstepcounter{qwerty}\theqwerty\label{seq:ref5}}. Построить график функции  $x(t)=3\cdot \cos (t),y(t)=2\cdot
\sin (t)$ .

График задан в параметрической форме и представляет собой эллипс. Выберем интервал построения графика  $[0;2\cdot \pi
]$, ранжируем переменную \textstyleEmphasis{t} на этом интервале, сформируем массивы \textstyleEmphasis{x} и
\textstyleEmphasis{y} и построим точеmчный график. Текст программы и результаты ее работы (рис. \ref{seq:ref5a})
представлены далее.

{\upshape
\#include {\textless}mgl2/qt.h{\textgreater}}

{\upshape
\#include {\textless}iostream{\textgreater}}

{\upshape
\#include {\textless}math.h{\textgreater}}

{\upshape
using namespace std;}

{\upshape
int sample(mglGraph *gr)}

{\upshape
\{\ \ //График эллипса}

{\upshape
\ \ int i,n; float h,a,b,t;}

{\upshape
\ \ a=0;}

{\upshape
\ \ b=2*M\_PI;}

{\upshape
\ \ n=200;}

{\upshape
\ \ h=(b-a)/n;}

{\upshape
\ \ //Формирование массивов абсцисс и ординат}

{\upshape
\ \ mglData x(n),y(n);}

{\upshape
\ \ for(i=0;i{\textless}n;i++)}

{\upshape
\ \ \{}

{\upshape
\ \ \ \ t=a+i*h;}

{\upshape
\ \ \ \ x[i]=3*cos(t);}

{\upshape
\ \ \ \ y[i]=2*sin(t);}

{\upshape
\ \ \}}

{\upshape
\ \ //Границы по осям координат}

{\upshape
\ \ gr-{\textgreater}SetRanges(-3,3,-2,2);}

{\upshape
\ \ //Оси координат}

{\upshape
\ \ gr-{\textgreater}Axis();}

{\upshape
\ \ //Сетка}

{\upshape
\ \ gr-{\textgreater}Grid();}

{\upshape
\ \ //График функции}

{\upshape
\ \ gr-{\textgreater}Plot(x,y,null{<<}knull{<<});}

{\upshape
\ \ gr-{\textgreater}SetFontSize(2);}

{\upshape
\ \ gr-{\textgreater}Title(null{<<}График эллипсаnull{<<});}

{\upshape
\ \ gr-{\textgreater}SetFontSize(4);}

{\upshape
\ \ gr-{\textgreater}Label('x',null{<<}OXnull{<<},0);}

{\upshape
\ \ gr-{\textgreater}Label('y',null{<<}OYnull{<<});}

{\upshape
return 0;}

{\upshape
\}}

{\upshape
int main(int argc,char **argv)}

{\upshape
\{}

{\upshape
\ \ setlocale(LC\_CTYPE, null{<<}ru\_RU.utf8null{<<});}

{\upshape
\ \ mglQT gr(sample,null{<<}Plotnull{<<});}

{\upshape
\ \ return gr.Run();}

{\upshape
\}}

\begin{minipage}{11.317cm}
{\itshape
Рис. {\refstepcounter{qwertya}\theqwertya\label{seq:ref5a}}: График к задаче \ref{seq:ref5}}
\includegraphics[scale=0.33]{mgl-img007}\end{minipage}

ЗАДАЧА {\refstepcounter{qwerty}\theqwerty\label{seq:ref6}}. Построить график функции:

{\centering
 $z(x,y)=0.6\cdot \sin (2\cdot \pi \cdot x)\cdot \cos (3\cdot \pi \cdot y)+0.4\cdot \cos (3\cdot \pi \cdot x\cdot y)$ .
\par}

В данном случае необходимо построить график функции двух аргументов. Для этого нужно сформировать матрицу \textit{z} при
изменении значений аргументов \textit{x} и \textit{y} и отобразить полученные результаты.

Далее приведен текст программы и результаты ее работы (рис. \ref{seq:ref6a}).

{\upshape
\#include {\textless}mgl2/qt.h{\textgreater}}

{\upshape
\#include {\textless}iostream{\textgreater}}

{\upshape
using namespace std;}

{\upshape
int sample(mglGraph *gr)}

{\upshape
\{}

{\upshape
\ \ //Изображение поверхности}

{\upshape
\ \ //Диапазон изменения x, y, z.}

{\upshape
\ \ gr-{\textgreater}SetRanges(-5,5,-5,5,-1,2);}

{\upshape
\ \ //Размер матрицы z по х и по y}

{\upshape
\ \ mglData z(500,400);}

{\upshape
\ \ //Формирование матрицы z.}

{\upshape
\ \ z.Modify(null{<<}0.6*sin(2*pi*x)*sin(3*pi*y) + 0.4*cos(3*pi*(x*y))null{<<});}

{\upshape
\ \ //Вращение осей}

{\upshape
\ \ gr-{\textgreater}Rotate(40,60);}

{\upshape
\ \ gr-{\textgreater}Box(); gr-{\textgreater}Axis(); gr-{\textgreater}Grid();}

{\upshape
\ \ //График функции}

{\upshape
\ \ gr-{\textgreater}Mesh(z);}

\}

{\upshape
int main(int argc,char **argv)}

{\upshape
\{}

{\upshape
\ \ setlocale(LC\_CTYPE, null{<<}ru\_RU.utf8null{<<});}

{\upshape
\ \ mglQT gr(sample,null{<<}MathGL examplenull{<<});}

{\upshape
\ \ return gr.Run();}

\}

ЗАДАЧА {\refstepcounter{qwerty}\theqwerty\label{seq:ref7}}. Построить эллипсоид.

Текст программы и результаты её работы (рис. \ref{seq:ref7a}).

{\upshape
\#include {\textless}mgl2/qt.h{\textgreater}}

{\upshape
\#include {\textless}mgl2/mgl.h{\textgreater}}

{\upshape
\#include {\textless}iostream{\textgreater}}

{\upshape
using namespace std;}

{\upshape
int sample(mglGraph *gr)}

{\upshape
\{\ \ gr-{\textgreater}Title(null{<<}Эллипсоидnull{<<});}

{\upshape
\ \ mglData x(50,40),y(50,40),z(50,40);}

{\upshape
\ \ gr-{\textgreater}Fill(x,null{<<}0.1+0.8*sin(2*pi*x)*sin(2*pi*y)null{<<});}

{\upshape
\ \ gr-{\textgreater}Fill(y,null{<<}0.15+0.7*cos(2*pi*x)*sin(2*pi*y)null{<<});}

{\upshape
\ \ gr-{\textgreater}Fill(z,null{<<}0.2+0.6*cos(2*pi*y)null{<<});}

{\upshape
\ \ gr-{\textgreater}Rotate(50,60);}

{\upshape
\ \ gr-{\textgreater}Box();}

{\upshape
\ \ gr-{\textgreater}Surf(x,y,z,null{<<}BbwrRnull{<<});}

{\upshape
\ \ return 0;}

{\upshape
\}}

{\upshape
int main(int argc,char **argv)}

{\upshape
\{\ \ setlocale(LC\_CTYPE, null{<<}ru\_RU.utf8null{<<});}

{\upshape
\ \ mglQT gr(sample,null{<<}Ellipsenull{<<});}

{\upshape
\ \ return gr.Run();}

\}

{\centering \begin{minipage}{12.127cm}
{\itshape
Рис. {\refstepcounter{qwertya}\theqwertya\label{seq:ref6a}}: График к задаче \ref{seq:ref6}}
\includegraphics[scale=0.33]{mgl-img008}\end{minipage}\par}

{\centering \begin{minipage}{12.056cm}
{\itshape
Рис. {\refstepcounter{qwertya}\theqwertya\label{seq:ref7a}}: График к задаче \ref{seq:ref7}}
\includegraphics[scale=0.33]{mgl-img009}\end{minipage}%
%{\textless}!{}-{}-[if supportFields{]}{\textgreater}{\textless}span style={}'font{}-size:10.0pt;font{}-family:null{<<}Times New Romannull{<<},null{<<}serifnull{<<};
%mso{}-fareast{}-font{}-family:null{<<}Times New Romannull{<<};mso{}-ansi{}-language:RU;mso{}-fareast{}-language:
%RU;mso{}-bidi{}-language:AR{}-SA{}'{\textgreater}{\textless}span style={}'mso{}-element:field{}-end{}'{\textgreater}{\textless}/span{\textgreater}{\textless}/span{\textgreater}{\textless}![endif{]}{}-{}-{\textgreater}
%12 Февраль 2013 г. 20:27
\par}

В завершении приведём решение реальной инженерной задачи с использованием MathGL.

ЗАДАЧА {\refstepcounter{qwerty}\theqwerty\label{seq:ref8}}. .В «Основах химии» Д.И. Менделеева приводятся данные о
растворимости азотнокислого натрия  $\mathit{NaNO}_{3}$  в зависимости от температуры воды. Число условных частей 
$\mathit{NaNO}_{3}$ растворяющихся в 100 частях воды при соответствующих температурах представлено в таблице. 

\begin{center}
\tablefirsthead{}
\tablehead{}
\tabletail{}
\tablelasttail{}
\begin{supertabular}{|m{1.573cm}|m{1.573cm}|m{1.573cm}|m{1.573cm}|m{1.573cm}|m{1.573cm}|m{1.573cm}|m{1.573cm}|m{1.574cm}|}
\hline
0 &
4 &
10 &
15 &
21 &
29 &
36 &
51 &
68\\\hline
66.7 &
71.0 &
76.3 &
80.6 &
85.7 &
92.9 &
99.4 &
113.6 &
125.1\\\hline
\end{supertabular}
\end{center}
Требуется определить растворимость азотнокислого натрия при температурах 25, 32 и 45 градусах в случае линейной
зависимости и найти коэффициент корреляции.

Параметры линейной зависимости (линии регрессии) $y=a+\mathit{bx}$ подбираются методом наименьших квадратов и
рассчитываются по формулам  $\left\{\genfrac{}{}{0pt}{0}{a=\frac{\sum _{i=1}^{n}y_{i}-b\cdot \sum
_{i=1}^{n}x_{i}}{n}}{b=\frac{n\cdot \sum _{i=1}^{n}{y_{i}\cdot x_{i}-\sum _{i=1}^{n}y_{i}\cdot \sum
_{i=1}^{n}x_{i}}}{n\sum _{i=1}^{n}{x_{i}^{2}}-\left(\sum _{i=1}^{n}x_{i}\right)^{2}}}\right.$.

Этапы решения задачи:

\begin{enumerate}
\item \begin{enumerate}
\item Ввод исходных данных из текстового файла.
\item Вычисление параметров a и b.
\item Расчёт значений линейной зависимости  $y=a+bx$ в точках  25, 32, 45.
\item Изображение графиков: экспериментальных точек, линии регрессии и рассчитанных значений.
\end{enumerate}
\end{enumerate}
Исходные данные задачи хранятся в текстовом файле input.txt (см. рис. \ref{seq:ref9}). 

\begin{minipage}{11.88cm}
{\itshape
Рис. {\refstepcounter{qwertya}\theqwertya\label{seq:ref9}}. Файл исходных данных к задаче \ref{seq:ref8}}
\includegraphics[scale=0.33]{mgl-img010}\end{minipage}

В первой строке файла хранится количество экспериментальных точек, в следующих двух строках – массивы абсцисс и ординат
экспериментальных точек. В четвёртой строке хранится количество (3) и точки (25, 32, 45), в которых необходимо
вычислить ожидаемое значение.

Текст программы решения задачи с комментариями приведен ниже.

{\upshape
\#include {\textless}mgl2/qt.h{\textgreater}}

{\upshape
\#include {\textless}iostream{\textgreater}}

{\upshape
\#include {\textless}fstream{\textgreater}}

{\upshape
using namespace std;}

{\upshape
int sample(mglGraph *gr)}

{\upshape
\{\ \ mglData x2(70),y2(70) ;}

{\upshape
\ \ int i,n,k; }

{\upshape
\ \ float h,a,b,sx=0,sy=0,syx=0,sx2=0;}

{\upshape
//Поток для чтения файла исходных данных}

{\upshape
\ \ ifstream F;}

{\upshape
F.open(null{<<}input.txtnull{<<});}

{\upshape
//Чтение исходных данных}

{\upshape
F{\textgreater}{\textgreater}n;}

{\upshape
//n- количество экспериментальных точек.}

{\upshape
// x(n),y(n) – координаты экспериментальных точек}

{\upshape
mglData x(n),y(n);}

{\upshape
cout{\textless}{\textless}null{<<}X{\textbackslash}nnull{<<};}

{\upshape
\ \ for(i=0;i{\textless}n;i++)}

{\upshape
\ \ \{F{\textgreater}{\textgreater}x[i];}

{\upshape
\ \ cout{\textless}{\textless}x[i]{\textless}{\textless}null{<<} null{<<};\}}

{\upshape
\ \ cout{\textless}{\textless}endl;}

{\upshape
cout{\textless}{\textless}null{<<}Y{\textbackslash}nnull{<<};}

{\upshape
\ \ for(i=0;i{\textless}n;i++)}

{\upshape
\ \ \{F{\textgreater}{\textgreater}y[i];}

{\upshape
\ \ cout{\textless}{\textless}y[i]{\textless}{\textless}null{<<} null{<<};\}}

{\upshape
\ \ cout{\textless}{\textless}endl;}

{\upshape
F{\textgreater}{\textgreater}k;}

{\upshape
cout{\textless}{\textless}null{<<}k=null{<<}{\textless}{\textless}k{\textless}{\textless}endl;}

{\upshape
mglData xr(k),yr(k);}

{\upshape
//xr, yr – ожидаемые значения, }

{\upshape
\ \ for(i=0;i{\textless}k;i++)}

{\upshape
\ \ F{\textgreater}{\textgreater}xr[i];}

{\upshape
\ \ cout{\textless}{\textless}endl;}

{\upshape
\ \ for(i=0;i{\textless}n;i++)}

{\upshape
\ \ \{}

{\upshape
\ \ \ \ sx+=x[i];}

{\upshape
\ \ \ \ sy+=y[i];}

{\upshape
\ \ \ \ syx+=y[i]*x[i];}

{\upshape
\ \ \ \ sx2+=x[i]*x[i];}

{\upshape
\ \ \}}

{\upshape
//Расчёт коэффициентов линии регрессиии.}

{\upshape
\ \ b=(n*syx-sy*sx)/(n*sx2-sx*sx);}

{\upshape
\ \ a=(sy-b*sx)/n;}

{\upshape
\ \ cout{\textless}{\textless}null{<<}a=null{<<}{\textless}{\textless}a{\textless}{\textless}null{<<}
b=null{<<}{\textless}{\textless}b{\textless}{\textless}endl;}

{\upshape
//Формирование массивов для изображения линии регрессии на}

{\upshape
// графике.}

{\upshape
\ \ for(i=0;i{\textless}70;i++)}

{\upshape
\ \ \{}

{\upshape
\ \ \ \ x2[i]=x[i]+1;}

{\upshape
\ \ \ \ y2[i]=a+b*x2[i];}

{\upshape
\ \ \}}

{\upshape
//Вычисление ожидаемых значений – растворимость}

{\upshape
//азотнокислого натрия при температурах 25, 32 и 45}

{\upshape
// градусах .}

{\upshape
\ \ cout{\textless}{\textless}null{<<}Xr  Yr{\textbackslash}nnull{<<};}

{\upshape
\ \ for(i=0;i{\textless}k;i++)}

{\upshape
\ \ \{yr[i]=a+b*xr[i];}

{\upshape
\ \ cout{\textless}{\textless}xr[i]{\textless}{\textless}null{<<}
null{<<}{\textless}{\textless}yr[i]{\textless}{\textless}endl;\}}

{\upshape
\ \ gr-{\textgreater}SetRanges(x[0],80,70,140);\ \ }

{\upshape
\ \ gr-{\textgreater}SetFontSize(3);}

{\upshape
\ \ //Оси координат}

{\upshape
\ \ gr-{\textgreater}Axis();}

{\upshape
\ \ //Сетка}

{\upshape
\ \ gr-{\textgreater}Grid();}

{\upshape
\ \ //Первая легенда}

{\upshape
\ \ gr-{\textgreater}AddLegend(null{<<}Экспериментnull{<<},null{<<}b onull{<<});}

{\upshape
\ \ //График экспериментальных точек, голубой (b) цвет.}

{\upshape
\ \ gr-{\textgreater}Plot(x,y,null{<<}b onull{<<});}

{\upshape
\ \ //Вторая легенда}

{\upshape
\ \ gr-{\textgreater}AddLegend(null{<<}Расчетnull{<<},null{<<}r *null{<<});}

{\upshape
\ \ //График ожидаемых значений.}

{\upshape
\ \ gr-{\textgreater}Plot(xr,yr,null{<<}r *null{<<});}

{\upshape
\ \ //Третья легенда}

{\upshape
\ \ gr-{\textgreater}AddLegend(null{<<}Линия регрессииnull{<<},null{<<}k-null{<<});}

{\upshape
\ \ //Изображение линиии регрессии.}

{\upshape
\ \ gr-{\textgreater}Plot(x2,y2,null{<<}k-null{<<});}

{\upshape
\ \ //Заголовок}

{\upshape
\ \ gr-{\textgreater}Title(null{<<}Задача Менделееваnull{<<});}

{\upshape
\ \ //Подпись по оси абсцисс}

{\upshape
\ \ gr-{\textgreater}Label('x',null{<<}tnull{<<},0);}

{\upshape
\ \ //Подпись по оси ординат}

{\upshape
\ \ gr-{\textgreater}Label('y',null{<<}NaNO\_3null{<<});}

{\upshape
\ \ //Вывод легенды}

{\upshape
\ \ gr-{\textgreater}Legend(2);}

{\upshape
return 0;}

{\upshape
\}}

{\upshape
int main(int argc,char **argv)}

{\upshape
\{}

{\upshape
\ \ setlocale(LC\_CTYPE, null{<<}ru\_RU.utf8null{<<});}

{\upshape
\ \ mglQT gr(sample,null{<<}Plotnull{<<});}

{\upshape
\ \ return gr.Run();}

{\upshape
\}}

После запуска программы на экране пользователь увидит следующие значения

{\bfseries
X\textlatin{[D?]}}

{\bfseries
0 4 10 15 21 29 36 51 68 \textlatin{[D?]}}

{\bfseries
Y\textlatin{[D?]}}

{\bfseries
66.7 71 76.3 80.6 85.7 92.9 99.4 113.6 125.1 \textlatin{[D?]}}

{\bfseries
k=3\textlatin{[D?]}}

{\bfseries
a=67.5078 b=0.87064\textlatin{[D?]}}

{\bfseries
Xr  Yr\textlatin{[D?]}}

{\bfseries
25 89.2738\textlatin{[D?]}}

{\bfseries
32 95.3683\textlatin{[D?]}}

{\bfseries
45 106.687\textlatin{[D?]}}

Графическое решение задачи, полученное с помощью средств библиотеки MathGL, представлено на рис. \ref{seq:ref10}.

\begin{minipage}{15.956cm}
{\itshape
Рис. {\refstepcounter{qwertya}\theqwertya\label{seq:ref10}}. Графическое решение задачи \ref{seq:ref8}}
\includegraphics[scale=0.33]{mgl-img011}\end{minipage}

Для изучения всех возможностей MathGL, авторы советуют обратиться к документации по MathGL. При освоении библиотеки
MathGL следует помнить, что логика и синтаксис библиотеки напоминает синтаксис Scilab и  Octave.

\clearpage\printindex
%
%{\textless}!{}-{}-EndFragment{}-{}-{\textgreater}
%12 Февраль 2013 г. 20:27

\endinput
