\chapter[Знакомство с языком \Sys{С++}]{Знакомство с языком \Sys{С++}}

В этой главе читатель напишет свои первые программы на языке \Sys{С(\Sys{С++})}, 
познакомится с основными этапами перевода программы
с языка \Sys{C++} в машинный код. Второй параграф главы посвящён 
знакомству со средой \Sys{Qt Creator}.

\section[Первая программа на \Sys{C++}]{Первая программа на \Sys{C++}}
Знакомство с языком \Sys{С++} начнём с написания программ, предназначенных 
для решения нескольких несложных задач.

ЗАДАЧА {\refstepcounter{qwerty}\theqwerty\label{seq:ref0}}. Заданы две стороны прямоугольника a, b. Найти его площадь и
периметр.

Как известно, периметр прямоугольника равен, а его площадь вычисляется по формуле $S=a\dot{b}$. 

Ниже приведён текст программы. 

\#include {\textless}iostream{\textgreater} 

using namespace std; 

int main() 

\{ 

\ \ float a,b,s,p; 

\ \ cout{\textless}{\textless}null{<<}a=null{<<}; 

\ \ cin{\textgreater}{\textgreater}a; 

\ \ cout{\textless}{\textless}null{<<}b=null{<<}; 

\ \ cin{\textgreater}{\textgreater}b; 

\ \ p=2*(a+b); 

\ \ s=a*b; 

\ \ cout {\textless}{\textless} null{<<}Периметр прямоугольника равен null{<<} {\textless}{\textless} p
{\textless}{\textless}endl; 

\ \ cout {\textless}{\textless} null{<<}Площадь прямоугольника равна null{<<} {\textless}{\textless} s
{\textless}{\textless}endl; 

\ \ return 0; 

\}

Давайте построчно рассмотрим текст программы и познакомимся со структурой программы на \Sys{С++} и с некоторыми операторами
языка. 

\textbf{Строка 1.} Указывает компилятору (а точнее, препроцессору), что надо использовать функции из стандартной
библиотеки \index{Библиотека!iostream}iostream. Библиотека \textstyleExample{iostream} нужна для организации ввода с
помощью инструкции \textstyleExample{cin} и вывода – с помощью \textstyleExample{cout}. В программе на языке \Sys{C++} должны
быть подключены все используемые библиотеки.

\textbf{Строка 2.} Эта строка обозначает, что при вводе и выводе с помощью \textstyleExample{cin} и
\textstyleExample{cout} будут использоваться стандартные устройства (клавиатура и экран), если эту строку не указывать,
то каждый раз при вводе вместо \textstyleExample{cin} надо будет писать \textstyleExample{std::cin}, а вместо
\textstyleExample{cout – std::cout}.

\textbf{Строка 3.} Заголовок главной функции (главная функция имеет имя \textstyleExample{main}). В простых программах
присутствует только функция \textstyleExample{main()}.

\textbf{Строка 4.} Любая функция начинается с символа \textstyleExample{\{}.

\textbf{Строка 5.} Описание вещественных (\textstyleExample{float}) переменных \textstyleExample{a} (длина одной стороны
прямоугольника), \textstyleExample{b} (длина второй стороны прямоугольника), \textstyleExample{s} (площадь
прямоугольника), \textstyleExample{p} (периметр прямоугольника). \textstyleEmphasis{Имя переменной}\footnote{В 
литературе равнозначно используются термины «\textstyleEmphasis{имя переменной}» и
«\textstyleEmphasis{идентификатор}».} состоит из латинских букв, цифр и символа подчёркивания. Имя не может начинаться
с цифры. В языке \Sys{С++} большие и малые буквы различимы. Например, имена \textstyleExample{PR\_1},
\textstyleExample{pr\_1}, \textstyleExample{Pr\_1} и \textstyleExample{pR\_1} – разные.

\textbf{Строка 6.} Вывод строки символов \textstyleExample{a=} с помощью \textstyleExample{cout}. Программа выведет
подсказку пользователю, что необходимо вводить переменную \textstyleExample{a}

\textbf{Строка 7.} Ввод вещественного числа \textstyleExample{a} с помощью \textstyleExample{cin}. В это момент
программа останавливается и ждёт, пока пользователь введёт значение переменой \textstyleExample{a} с клавиатуры.

\textbf{Строка 8}. Вывод строки символов \textstyleExample{b=} с помощью \textstyleExample{cout}.

\textbf{Строка 9.} Ввод вещественного числа \textstyleExample{b} с помощью \textstyleExample{cin}.

\textbf{Строка 10.} Оператор присваивания для вычисления периметра прямоугольника (переменная \textstyleExample{p}) по
формуле  $2\dot{(a+b)}$ . В операторе присваивания могут использоваться круглые скобки и знаки операций: + (сложение),
- (вычитание), * (умножение), / (деление).

\textbf{Строка 11.} Оператор присваивания для вычисления площади прямоугольника.

\textbf{Строка 12.} Вывод строки «Периметр прямоугольника равен » и значения \textstyleExample{p} на экран. Константа
\textstyleExample{endl} хранит строку null{<<}\textstyleExample{{\textbackslash}nnull{<<}}, которая предназначена для
перевода курсора в новую строку дисплея\footnote{Обращаем внимание читателя, что символ пробел является обычным
символом, который ничем не отличается от остальных. Для вывода пробела на экран его надо явно указывать в строке
вывода.}. Таким образом строка 

cout {\textless}{\textless} null{<<}Периметр прямоугольника равен null{<<} {\textless}{\textless} p
{\textless}{\textless}endl; 

выводит на экран текст null{<<}\textstyleExample{Периметр прямоугольника равен null{<<}}\footnote{С пробелом после слова
«равен».}, значение переменной \textstyleExample{p}, и переводит курсор в новую строку.

\textbf{Строка 13.} Вывод строки null{<<}\textstyleExample{Площадь прямоугольника равна} null{<<}, значения площади
прямоугольника \textstyleExample{s}, после чего курсор переводится в новую строку дисплея.

\textbf{Строка 14.}\textbf{\textmd{ Оператор }}\textstyleExample{return}\textbf{\textmd{, который возвращает значение в
операционную систему. Об этом подробный разговор предстоит в п. 4.9. Сейчас следует запомнить, если программа
начинается со строки }}\textstyleExample{int main()}\textbf{\textmd{, последним оператором должен быть
}}\textstyleExample{return 0}\footnote{Вообще говоря, вместо 0 может быть любое целое число.}\textbf{\textmd{.}}

\textbf{Строка 15.} Любая функция (в том числе и \textstyleExample{main}) заканчивается символом \textstyleExample{\}}.

Мы рассмотрели простейшую программу на языке \Sys{С++}, состоящую из операторов ввода данных, операторов присваивания (в
которых происходит расчет по формулам) и операторов вывода. 

Любая программа на языке \Sys{С++} представляет собой одну или несколько функций. В любой программе \textbf{обязательно}
должна быть одна функция \index{Функция!main}main(). C этой функции начинается выполнение программы. Правилом хорошего
тона в программировании является разбиение задачи на подзадачи, и в главной функции чаще всего должны быть операторы
вызова других функций. Общую структуру любой \index{Структура программы}программы на языке \Sys{C++} можно записать следующим
образом. 

Директивы препроцессора

Объявление глобальных переменных

Тип\_результата f1(Список\_переменных)

\{

Операторы

\}

Тип\_результата f2(Список\_переменных)

\{

Операторы

\}

...

Тип\_результата fn(Список\_переменных)

\{

Операторы

\}

Тип\_ результата main(Список\_переменных)

\{

Операторы

\}

На первом этапе знакомства с языком мы будем писать программы, состоящие только из функции main, без использования
глобальных переменных. Структура самой  простой на С(\Sys{С++}) имеет вид.

Директивы препроцессора

Тип\_ результата main(Список\_переменных)

\{

Операторы

\}

Введенная в компьютер программа на языке \Sys{С++} должна быть переведена в двоичный машинный код (должен быть сформирован
исполняемый файл). Для этого существуют специальные программы, называемые трансляторами. Все
\index{транслятор}\index{транслятор}трансляторы  \textit{ }делятся на два класса:

\begin{itemize}
\item \index{интерпретатор}\textit{интерпретатор}\textit{ы}%
%{\textless}!{}-{}-[if supportFields{]}{\textgreater}{\textless}span
%style={}'mso{}-element:field{}-begin{}'{\textgreater}{\textless}/span{\textgreater}XE \&quot;Интерпретатор\&quot; {\textless}![endif{]}{}-{}-{\textgreater}
%12 Февраль 2013 г. 20:27
\textit{ }– трансляторы, которые переводят каждый оператор программы в машинный код, и по мере перевода операторы
выполняются процессором; 
\item \index{компилятор}\textit{компилятор}\textit{ы}%
%{\textless}!{}-{}-[if supportFields{]}{\textgreater}{\textless}span
%style={}'mso{}-element:field{}-begin{}'{\textgreater}{\textless}/span{\textgreater}XE \&quot;Компилятор\&quot; {\textless}![endif{]}{}-{}-{\textgreater}
%12 Февраль 2013 г. 20:27
%
%{\textless}!{}-{}-[if supportFields{]}{\textgreater}{\textless}span
%style={}'mso{}-element:field{}-end{}'{\textgreater}{\textless}/span{\textgreater}{\textless}![endif{]}{}-{}-{\textgreater}
%12 Февраль 2013 г. 20:27
 переводят всю программу целиком, и если перевод всей программы прошел без ошибок, то полученный двоичный код можно
запускать на выполнение. 
\end{itemize}
Процесс перевода программы в машинный код называется \textit{трансляцией}%
%{\textless}!{}-{}-[if supportFields{]}{\textgreater}{\textless}span
%style={}'mso{}-element:field{}-begin{}'{\textgreater}{\textless}/span{\textgreater}XE \&quot;Трансляция программы\&quot; {\textless}![endif{]}{}-{}-{\textgreater}
%12 Февраль 2013 г. 20:27
%
%{\textless}!{}-{}-[if supportFields{]}{\textgreater}{\textless}span
%style={}'mso{}-element:field{}-end{}'{\textgreater}{\textless}/span{\textgreater}{\textless}![endif{]}{}-{}-{\textgreater}
%12 Февраль 2013 г. 20:27
. Если в качестве транслятора выступает компилятор, то используют термин \textit{компиляция}%
%{\textless}!{}-{}-[if supportFields{]}{\textgreater}{\textless}span
%style={}'mso{}-element:field{}-begin{}'{\textgreater}{\textless}/span{\textgreater}XE \&quot;Компиляция программы\&quot; {\textless}![endif{]}{}-{}-{\textgreater}
%12 Февраль 2013 г. 20:27
%
%{\textless}!{}-{}-[if supportFields{]}{\textgreater}{\textless}span
%style={}'mso{}-element:field{}-end{}'{\textgreater}{\textless}/span{\textgreater}{\textless}![endif{]}{}-{}-{\textgreater}
%12 Февраль 2013 г. 20:27
~программы. При переводе программы с языка \Sys{С++} в машинный код используются именно компиляторы, и поэтому применительно к
языку \Sys{С++} термины «компилятор» и «транслятор» эквивалентны.

Рассмотрим основные этапы обработки компилятором программы на языке \Sys{С++} и формирования машинного кода.

\begin{enumerate}
\item Сначала программа обрабатывается препроцессором\footnote{Препроцессор – это программа, которая преобразовывает
текст директив препроцессора в форму, понятную компилятору. О данных на выходе препроцессора говорят, что они находятся
в препроцессированной форме.}, который обрабатывает директивы препроцессора, в нашем случае это директивы включения
заголовочных файлов (файлов с расширением \textbf{.h}) - текстовых файлов, в которых содержится описание используемых
библиотек. В результате формируется полный текст программы, который поступает на вход компилятора. 
\item Компилятор разбирает текст программ на составляющие элементы, проверяет синтаксические ошибки и в случае их
отсутствия формирует объектный код (файл с расширением \textbf{.o} или .\textbf{obj}). Получаемый на этом этапе
двоичный код не включает в себя двоичные коды библиотечных функций и функций пользователя.
\item \textit{Компоновщик}%
%{\textless}!{}-{}-[if supportFields{]}{\textgreater}{\textless}span
%style={}'mso{}-element:field{}-begin{}'{\textgreater}{\textless}/span{\textgreater}XE \&quot;Компоновщик\&quot; {\textless}![endif{]}{}-{}-{\textgreater}
%12 Февраль 2013 г. 20:27
%
%{\textless}!{}-{}-[if supportFields{]}{\textgreater}{\textless}span
%style={}'mso{}-element:field{}-end{}'{\textgreater}{\textless}/span{\textgreater}{\textless}![endif{]}{}-{}-{\textgreater}
%12 Февраль 2013 г. 20:27
~подключает к объектному коду программы объектные модули библиотек и других файлов (если программа состоит из нескольких
файлов) и генерирует исполняемый код программы (двоичный файл), который уже можно запускать на выполнение. Этот этап
называется компоновкой или сборкой программы.
\end{enumerate}
После написания программы ее необходимо ввести в компьютер. В той книге будет рассматриваться работа на языке \Sys{C++} в
среде \Sys{Qt Creator}\footnote{Тексты программ, приведённые в первой части книги (главы 1 — 9), без серьёзных изменений
могут быть откомпилированы с помощью любого современного компилятора с языка С(\Sys{С++}). Авторы протестировали все
программы из первой части книги с помощью  \Sys{QT Creator} и IDE Geany (с использованием g++ версии 4.8).}. Поэтому перед
вводом программы в компьютер надо познакомиться со средой программирования.

\subsection[Среда программирования \Sys{Qt Creator}]{Среда программирования \Sys{Qt Creator}}
\index{Среда программирования \Sys{Qt Creator}}Среда программирования \Sys{Qt Creator} (IDE \Sys{QT Creator}) находится в репозитории
большинства современных дистрибутивов Linux (OC Linux Debian, OC Linux Ubuntu, OC ROSA Linux, ALT Linux и др.).
Установка осуществляется штатными средствами вашей операционной системы (менеджер пакетов Synaptic и др.) из
репозитория, достаточно установить пакет qtcreator, необходимые пакеты и библиотеки будут доставлены  автоматически.
Последнюю версию IDE \Sys{Qt Creator} можно скачать на сайте QtProject (http://qt-project.org/downloads). Скачанный
установочный файл имеет расширениe \textbf{.run}. Для установки приложения, необходимо запустить его на выполнение.
Установка проходит в графическом режиме. После запуска программы пользователь увидит на экране окно, подобное
представленному на рис. \ref{seq:refDrawing0}\footnote{Окно на вашем компьютере визуально может несколько отличаться от
представленного на рис. \ref{seq:refDrawing0}, авторы использовали IDE \Sys{Qt Creator} версии 2.6.2, основанную на QT
5.0.1.}.

 [Warning: Image ignored] % Unhandled or unsupported graphics:
%\includegraphics[scale=0.33]{Glava1-img001}
\captionof{figure}[Окно \Sys{Qt Creator}]{Окно \Sys{Qt Creator}}
\label{seq:refDrawing0}


При работе в \Sys{Qt Creator} вы находитесь в одном из режимов:

\begin{enumerate}
\item \textbf{Welcome} (Начало) – отображает экран приветствия, позволяя быстро загружать недавние сессии или отдельные
проекты. Этот режим можно увидеть при запуске \Sys{Qt Creator} без указания ключей командной строки.
\item \textbf{Edi}t (Редактор) – позволяет редактировать файлы проекта и исходных кодов. Боковая панель слева
предоставляет различные виды для перемещения между файлами.
\item \textbf{Debug} (Отладка) – предоставляет различные способы для просмотра состояния программы при отладке;
\item \textbf{Projects} (Проекты) – используется для настройки сборки, запуска и редактирования кода.
\item \textbf{Analyze} (Анализ) – в Qt интегрированы современные средства анализа кода разрабатываемого приложения.
\item \textbf{Help} (Справка) – используется для вывода документации библиотеки Qt и \Sys{Qt Creator}.
\item \textbf{Output} (Вывод) – используется для вывода подробных сведений о проекте.
\end{enumerate}
Рассмотрим простейшие приёмы работы в среде \Sys{Qt Creator} на примере \index{Консольное приложение!создание}создания
консольного приложения для решения задачи \ref{seq:ref0}. Для этого можно поступить одним из способов:

\begin{enumerate}
\item В меню \textbf{File} (Файл) выбрать команду \textbf{New File or Project} (Новый файл или проект) (комбинация
клавиш \textbf{Ctrl+N}).
\item Находясь в режиме \textbf{Welcome} (Начало) главного окна QtCreator (рис. \ref{seq:refDrawing0}) щёлкаем по ссылке
\textbf{Develop} (Разработка) и выбираем команду Create Project (Создать проект).
\end{enumerate}
После это откроется окно, подобное представленному на рис. \ref{seq:refDrawing1}. Для создания простейшего консольного
приложения выбираем \textbf{Non-Qt Project} (Проект без использования Qt) – \textbf{Plain \Sys{C++} Project} (Простой проект
на языке \Sys{С++}).

Далее выбираем имя проекта и каталог для его размещения (см. рис. \ref{seq:refDrawing2})\footnote{Рекомендуем для
каждого (даже самого простого) проекта выбирать отдельный каталог. Даже самый простой проект – это несколько
взаимосвязанных между собой файлов и каталогов.}. Следующие два этапа создания нашего первого приложения оставляем без
изменения\footnote{О назначении этих этапов будет рассказано в дальнейших разделах книги.}. После чего окно IDE Qt
Creator примет вид, подобное, представленное на рис. \ref{seq:refDrawing3}. Заменим текст текста стандартного
приложения, которое выводит текст «Hello, Word», на текст программы решения задачи \ref{seq:ref0}.

\begin{figure}[htb]
\centering
 [Warning: Image ignored] % Unhandled or unsupported graphics:
%\includegraphics[scale=0.33]{Glava1-img002}
\caption[Окно выбора типа приложения в \Sys{Qt Creator}]{Окно выбора типа приложения в \Sys{Qt Creator}}
\label{seq:refDrawing1}

\end{figure}
 [Warning: Image ignored] % Unhandled or unsupported graphics:
%\includegraphics[scale=0.33]{Glava1-img003}
\captionof{figure}[Выбор имени и каталога нового проекта]{Выбор имени и каталога нового проекта}
\label{seq:refDrawing2}


Для сохранения текста программы можно воспользоваться \textbf{Сохранить} или \textbf{Сохранить всё} из меню
\textbf{Файл}. Откомпилировать и \index{Консольное приложение!запуск}запустить программу можно одним из следующих
способов:

 [Warning: Image ignored] % Unhandled or unsupported graphics:
%\includegraphics[scale=0.33]{Glava1-img004}
\captionof{figure}[Главное окно создания консольного приложения]{Главное окно создания консольного приложения}
\label{seq:refDrawing3}


\begin{enumerate}
\item Пункт меню \textbf{Сборка-Запустить}.
\item Нажать на клавиатуре комбинацию клавиш Ctrk+R.
\item Щёлкнуть по кнопке Запустить (  [Warning: Image ignored] % Unhandled or unsupported graphics:
%\includegraphics[scale=0.33]{Glava1-img005}
 ).
\end{enumerate}
\begin{enumerate}
\item[] Окно с результатами работы программы представлено на рис. \ref{seq:refDrawing4}.
\end{enumerate}
 [Warning: Image ignored] % Unhandled or unsupported graphics:
%\includegraphics[scale=0.33]{Glava1-img006}
\captionof{figure}[Результаты работы программы решения задачи 1.1]{Результаты работы программы решения задачи
\ref{seq:ref0}}
\label{seq:refDrawing4}


Авторы сталкивались с тем, что в некоторых дистрибутивах Ubuntu Linux и Linux Mint после установки \Sys{Qt Creator} не
запускались консольные приложения. Если читатель столкнулся с подобной проблемой, скорее всего надо корректно настроить
терминал, который отвечает за запуск приложений в консоли. Для этого вызываем команду Tools – Options – Environment
(см. рис. \ref{seq:refDrawing5}). Параметр \textbf{Terminal} (Терминал) должен быть таким же, как показано на рис.
\ref{seq:refDrawing5}. Проверьте установлен ли в Вашей системе пакет xterm, и при необходимости доставьте его. После
этого не должно быть проблем с запуском консольных приложений.

 [Warning: Image ignored] % Unhandled or unsupported graphics:
%\includegraphics[scale=0.33]{Glava1-img007}
\captionof{figure}[Окно настроек среды \Sys{Qt Creator}]{Окно настроек среды \Sys{Qt Creator}}
\label{seq:refDrawing5}


Аналогичным образом можно создавать и запускать любое консольное приложение.

Дальнейшее знакомство со средой \Sys{Qt Creator} продолжим, решая следующую задачу.

ЗАДАЧА \stepcounter{qwerty}{\theqwerty}. Заданы длины трёх сторон треугольника a, b и c (см. рис.
\ref{seq:refDrawing6}). Вычислить площадь и периметр треугольника

 [Warning: Image ignored] % Unhandled or unsupported graphics:
%\includegraphics[scale=0.33]{Glava1-img008}
\captionof{figure}[Треугольник]{Треугольник}
\label{seq:refDrawing6}


Для решения задачи нам понадобится формула вычисления периметра  $p=a+b+c$. Для вычисления площади можно воспользоваться
формулой Герона  $S=\sqrt{\frac{p}{2}\left(\frac{p}{2}-a\right)\left(\frac{p}{2}-b\right)\left(\frac{p}{2}-a\right)}$.

Решение задачи можно разбить на следующие этапы:

\begin{enumerate}
\item  Определение значений a, b и c (ввод величин a, b, c с клавиатуры в память компьютера).
\item  Расчет значений p и s по приведенным выше формулам.
\item Вывод p и s на экран дисплея.
\end{enumerate}
Ниже приведен текст программы. Сразу заметим, что в тексте могут встречаться строки, начинающие с двух наклонных (//),
являющиеся комментариями. \textit{Комментарии}%
%{\textless}!{}-{}-[if supportFields{]}{\textgreater}{\textless}span
%style={}'mso{}-element:field{}-end{}'{\textgreater}{\textless}/span{\textgreater}{\textless}![endif{]}{}-{}-{\textgreater}
%12 Февраль 2013 г. 20:27
~не являются обязательными элементами программы и ничего не сообщают компьютеру, они поясняют человеку, читающему текст
программы, назначение отдельных элементов программы. В книге комментарии будут широко использоваться для пояснения
отдельных участков программы.

{\upshape
\#include {\textless}iostream{\textgreater} }

{\upshape
\#include {\textless}math.h{\textgreater} }

{\upshape
using namespace std; }

{\upshape
int main() }

{\upshape
\{ }

{\upshape
 float a,b,c,s,p; }

{\upshape
 cout{\textless}{\textless}null{<<}Введите длины сторон треугольникаnull{<<}{\textless}{\textless}endl; }

{\upshape
//Ввод значений длин треугольника a,b,c.}

{\upshape
 cin{\textgreater}{\textgreater}a{\textgreater}{\textgreater}b{\textgreater}{\textgreater}c; }

{\upshape
//Вычисление периметра треугольника.}

{\upshape
 p=a+b+c; }

{\upshape
//Вычисление площади треугольника.}

{\upshape
 s=sqrt(p/2*(p/2-a)*(p/2-b)*(p/2-c));}

{\upshape
//Вывод на экран дисплея значений площади и периметра}

{\upshape
// треугольника. }

{\upshape
 cout{\textless}{\textless}null{<<}Периметр треугольника равен null{<<}{\textless}{\textless}p{\textless}{\textless}}

{\upshape
null{<<}, его площадь равна null{<<}{\textless}{\textless}s{\textless}{\textless}endl; }

{\upshape
 return 0; }

{\upshape
\}}

Кроме используемой в предыдущей программе библиотеки iostream, в строке 2 подключим библиотеку
\index{Библиотека!math.h}math.h, которая служит для использования математических функций языка С(\Sys{С++}). В данной
программе используется функция извлечения квадратного корня – \index{Функция!sqrt(x)}\textbf{sqrt(x)}. Остальные
операторы (ввода, вывода, вычисления значений переменных аналогичны используемым в предыдущей программе.

Таким образом, выше были рассмотрены самые простые программы (линейной структуры), которые предназначены для ввода
исходных данных расчёта по формулам и вывода результатов.

%
%{\textless}!{}-{}-[if supportFields{]}{\textgreater}{\textless}span style={}'font{}-size:10.0pt;font{}-family:null{<<}Times New Romannull{<<},null{<<}serifnull{<<};
%mso{}-fareast{}-font{}-family:null{<<}Times New Romannull{<<};mso{}-ansi{}-language:RU;mso{}-fareast{}-language:
%RU;mso{}-bidi{}-language:AR{}-SA{}'{\textgreater}{\textless}span style={}'mso{}-element:field{}-end{}'{\textgreater}{\textless}/span{\textgreater}{\textless}/span{\textgreater}{\textless}![endif{]}{}-{}-{\textgreater}
%12 Февраль 2013 г. 20:27


\printindex
%
%{\textless}!{}-{}-EndFragment{}-{}-{\textgreater}
%12 Февраль 2013 г. 20:27

\endinput
