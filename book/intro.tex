%\chapter{Программирование на языке С++ в среде \Sys{Qt} Creator}

%Алексеев Е.Р., Злобин Г.Г., Костюк Д.А., Чеснокова О.В., Чмыхало А.C.

\chapter*{Предисловие}
Книга, которую открыл читатель, является с одной стороны учебником по алгоритмизации и программированию на~\Sys{C++}, а с
другой --- пособием по разработке визуальных приложений в среде \Sys{Qt Creator}. В книге описаны среда программирования~\Sys{Qt}
Creator и редактор \Sys{Geany}. При чтении книги не требуется предварительного знакомства с программированием. 

В первой части книги (главы 1--9) на большом количестве примеров представлены методы построения программ на языке~\Sys{C++},
особое внимание уделено построению циклических программ, программированию с использованием функций, массивов, матриц и
указателей. 

Вторая часть книги (глава 10) посвящена объектно-ориентированному программированию на~\Sys{C++}. 

В третьей части книги (главы 11--15) читатель научится создавать кроссплатформенные визуальные приложения с помощью~\Sys{Qt}
Creator и познакомится с библиотекой классов~\Sys{Qt}.

В книге присутствуют задания для самостоятельного решения.

В приложениях описан текстовый редактор \Sys{Geany}, а также кросс\-платформенная библиотека MathGL предназначенная для построения
различных двух- и трёхмерных графиков.

Главы 1--9 написаны Е.\,Р.\,Алексеевым и О.\,В.\,Чесноковой. Автором раздела по объектно-ориентированному программированию
является Д.\,А.\,Костюк. Главы 11--15, посвящённые программированию с использованием инструментария \Sys{Qt}, написаны 
Г.\,Г.\,Злобиным и А.\,C.\,Чмыхало.

Авторы благодарят компанию ALT Linux (www.altlinux.ru) и лично Алексея Смирнова и Владимира Чёрного за  возможность
издать очередную книгу по свободному программному обеспечению.

%\subsection{Аннотация}
%Книга является учебником по алгоритмизации и программированию на С++ и пособием по разработке визуальных приложений в
%среде \Sys{QT Creator}. Также в книге описаны среда программирования \Sys{Qt} Creator, редактор \Sys{Geany}, кроссплатформенная билиотека
%построения графиков MathGL. При чтении книги не требуется предварительного знакомства с программированием. 
%
%Издание предназначено для студентов, аспирантов и преподавателей вузов, а также для всех, кто изучает программирование
%на С++ и осваивает кроссплатформенный инструментарий \Sys{Qt} для разработки программного обеспечения. 
