\chapter[Строки в языке \Sys{C++}]{Строки в языке \Sys{C++}}\label{ch08}
В главе дано общее представление о строках в \Sys{C++}. Описана их структура, способы инициализации, возможности ввода-вывода,
приведены примеры обработки строк и текстов.

\section[Общие сведения о строках в \Sys{C++}]{Общие сведения о строках в \Sys{C++}}\label{ch08:1}
\index{Строка}\emph{Строка} --- последовательность символов. Для работы с символами в языке \Sys{C++} предусмотрен
тип данных \Sys{char}. Если в выражении встречается одиночный символ, он должен быть заключен в одинарные
кавычки. При использовании в выражениях строка заключается в двойные кавычки. Признаком конца строки является нулевой
символ '\Sys{{\textbackslash}0}'. В \Sys{C++} строки можно описать с помощью массива символов (массив
элементов типа \Sys{char}), в массиве следует предусмотреть место для хранения признака конца строки
('\Sys{{\textbackslash}0}').

Например, 
\begin{lstlisting}
char s[25]; //`Описана строка из 25 символов.`
//`Элемент s[25] предназначен для хранения символа конца строки.`
char s[7]="`\Sys{Привет}`";//`Описана строка из 7 символов и ей присвоено значение.`
//`Определен массив из 3 строк по 25 байт в каждой.`
char m[3][25]={"`\Sys{Пример }`", "`\Sys{использования}`", "`\Sys{ строк}`"} 
\end{lstlisting}

Для работы со строками можно использовать указатели (\Sys{char *}). Адрес первого символа будет начальным
значением указателя.

Рассмотрим пример объявления и ввода строк.
\begin{lstlisting}
#include <iostream>
using namespace std;
int main()
{
char s2[20], *s3, s4[30]; //`Описываем 3 строки, s3 — указатель.`
cout<<"`\Sys{Введите строку:}`"<<endl;
cout<<"s2=";cin>>s2; //`Ввод строки s2.`
cout<<"`\Sys{Была введена строка:}`"<<endl;
cout<<"s2="<<s2<<endl;
s3=s4; //`Запись в s3 адреса строки s4. Теперь в указателях s3 и s4` 
       //`хранится один адрес.`
cout<<"`\Sys{Введите строку:}`"<<endl;
cout<<"s3="; cin>>s3; //`Ввод строки s3.`
cout<<"`\Sys{Была введена строка:}`"<<endl;
cout<<"s3="<<s3<<endl; //`Вывод на экран s3 и s4,`
cout<<"`\Sys{Сформирована новая строка:}`"<<endl;
cout<<"s4="<<s4<<endl;//`s3 и s4 --- одно и тоже.`
return 0;
}
\end{lstlisting}

Если запустить эту программу на выполнение, то в консольном окне приложения будет получен следующий результат.
\begin{verbatim}
Введите строку:
s2=Привет!
Была введена строка:
s2=Привет!
Введите строку:
s3=Программируем?
Была введена строка:
s3=Программируем?
Сформирована новая строка:
s4=Программируем?
\end{verbatim}
Однако, если во вводимых строках появятся пробелы, программа будет работать не так, как ожидает пользователь:
\begin{verbatim}
Введите строку:
s2=Привет, Вася!
Была введена строка:
s2=Привет,
Введите строку:
s3=Была введена строка:
s3=Вася!
Сформирована новая строка:
s4=Вася!
\end{verbatim}
Дело в том, что функция \Sys{cin} вводит строки до встретившегося пробела. Более универсальной функцией
являет функция

\Sys{cin.getline(char *s, int n);}

она предназначена для \index{Строки!ввод}\emph{ввода с клавиатуры строки} \Sys{s} с пробелами,
причем в строке не должно быть более \Sys{n} символов. Например,
\begin{lstlisting}
char s[20];
cout<<"`\Sys{Введите строку:}`"<<endl;
cout<<"s2="; cin.getline(s,20);
cout<<"`\Sys{Была введена строка:}`"<<endl;
cout<<"s2="<<s2<<endl;
\end{lstlisting}

Результат:
\begin{verbatim}
Введите строку:
s2=Привет, Вася!
Была введена строка:
s2=Привет, Вася!
\end{verbatim}

\section[Операции над строками]{Операции над строками}\label{ch08:2}
Строку можно обрабатывать как массив символов, используя алгоритмы обработки массивов, или с помощью специальных
\index{Строки!функции обработки}\emph{функций обработки строк},
некоторые из которых приведены в таблицах~\ref{ch08:refTable0}--\ref{ch08:refTable1}. 

{\tabcolsep=0.1em\noindent\footnotesize
\begin{longtable}{|p{0.2\textwidth}|p{0.27\textwidth}|p{0.26\textwidth}|p{0.24\textwidth}|}
\caption{Функции работы со строками, библиотека \Sys{string.h}}\label{ch08:refTable0}\\
\hline
\Emph{Прототип функции} &\Emph{Описание функции} &\Emph{Пример использования} &\Emph{Результат}\\
\hline \hline
\endfirsthead
\multicolumn{4}{c}%
{{\tablename\ \thetable{} --- продолжение}} \\
\hline
\Emph{Прототип функции} &\Emph{Описание функции} &\Emph{Пример использования} &\Emph{Результат}\\
\hline \hline
\endhead
\lstinline!size_t strlen(const char *s)!& Вычисляет длину строки \Sys{s} в байтах &
\lstinline!char s[80];!\linebreak
\lstinline!cout<<"s=";!\linebreak
\lstinline!cin.getline(s,80);!\linebreak
\lstinline!cout<<"s="<<s<<endl<<"!\Sys{Длина строки}\lstinline!\t"<<strlen(s)<<endl;!&
\begin{verbatim}
s=Hello, Russia!
s=Hello, Russia! 
Длина строки 14
\end{verbatim}
\\\hline
\lstinline!char *strcat(char *dest, const char *scr)! & Присоединяет строку \Sys{src} в конец строки \Sys{dest}, 
полученная строка возвращается в качестве  результата &
\lstinline!char s1[80],s2[80];!\linebreak
\lstinline!cout<<"s1=";!\linebreak
\lstinline!cin.getline(s1,80);!\linebreak
\lstinline!cout<<"s2=";!\linebreak
\lstinline!cin.getline(s2,80);!\linebreak
\lstinline!cout<<"s="<<strcat(s1,s2);!
&
\begin{verbatim}
s1=Hello, 
s2=Russia!
s=Hello, Russia!
\end{verbatim}
\\\hline
\lstinline!char *strcpy(char *dest, const char *scr)!& Копирует строку \Sys{src} в место памяти, на которое указывает \Sys{dest} &
\lstinline!char s1[80],s2[80];!\linebreak
\lstinline!cout<<"s1=";!\linebreak
\lstinline!cin.getline(s1,80);!\linebreak
\lstinline!strcpy(s2,s1);!\linebreak
\lstinline!cout<<"s2="<<s2;!
&
\begin{verbatim}
s1=Hello,Russia!
s2=Hello,Russia!
\end{verbatim}
\\\hline
\lstinline!char *strncat(char *dest, const char *dest, size_t maxlen)!&Присоединяет строку \Sys{maxlen} символов строки \Sys{src} в конец строки \Sys{dest} &
\lstinline!char s1[80],s2[80];!\linebreak
\lstinline!cout<<"s1=";!\linebreak
\lstinline!cin.getline(s1,80);!\linebreak
\lstinline!cout<<"s2=";!\linebreak
\lstinline!cin.getline(s2,80);!\linebreak
\lstinline!cout<<"s="<<strncat(s1,s2,6);!&
\begin{verbatim}
s1=Hello, 
s2=Russia!
s=Hello, Russia
\end{verbatim}
\\\hline
\lstinline!char *strncpy(char *dest, const char *scr, size_t maxlen)!&Копирует \Sys{maxlen} символов строки \Sys{src} в место памяти, на которое указывает \Sys{dest} &
\lstinline!char s1[80],s2[80];!\linebreak
\lstinline!cout<<"s1=";!\linebreak
\lstinline!cin.getline(s1,80);!\linebreak
\lstinline!strncpy(s2,s1,5);!\linebreak
\lstinline!cout<<"s2="<<s2;!&
\begin{verbatim}
s1=Hello,Russia!
s2=Hello
\end{verbatim}
\\\hline
\lstinline!int strcmp(const char *s1, const char *s2)!&
Сравнивает две строки в лексикографическом порядке с учетом различия прописных и строчных букв, функция возвращает 0,
если строки совпадают, возвращает $-1$, если \Sys{s1} располагается в упорядоченном по алфавиту порядке
раньше, чем \Sys{s2}, и 1 в противоположном случае. &
\lstinline!char s1[80],s2[80];!\linebreak
\lstinline!cout<<"s1=";!\linebreak
\lstinline!cin.getline(s1,80);!\linebreak
\lstinline!cout<<"s2=";!\linebreak
\lstinline!cin.getline(s2,80);!\linebreak
\lstinline!cout<<strcmp(s1,s2)<<endl;!&
\begin{verbatim}
s1=RUSSIA
s2=Russia
-1
\end{verbatim}
\\\hline
\lstinline!int strncmp(const char *s1, const char *s2, size_t maxlen)!
 &Сравнивает \Sys{maxlen} символов двух строк в лексикографическом порядке, функция возвращает 0, если строки
совпадают, возвращает -1, если \Sys{s1} располагается в упорядоченном по алфавиту порядке раньше, чем
\Sys{s2}, и 1 --- в  противоположном случае. &
\lstinline!char s1[80],s2[80];!\linebreak
\lstinline!cout<<"s1=";!\linebreak
\lstinline!cin.getline(s1,80);!\linebreak
\lstinline!cout<<"s2=";!\linebreak
\lstinline!cin.getline(s2,80);!\linebreak
\lstinline!cout<<strncmp(s1,s2,6);!
&
\begin{verbatim}
s1=Hello,Russia!
s2=Hello,
0
\end{verbatim}
\\\hline
\end{longtable}
}


{\tabcolsep=0.1em\noindent\footnotesize
\begin{longtable}{|p{0.2\textwidth}|p{0.27\textwidth}|p{0.26\textwidth}|p{0.24\textwidth}|}
\caption{Функции работы со строками, библиотека \Sys{stdlib.h}}\label{ch08:refTable1}\\
\hline
\Emph{Прототип функции} &\Emph{Описание функции} &\Emph{Пример использования} &\Emph{Результат}\\
\hline \hline
\endfirsthead
\multicolumn{4}{c}%
{{\tablename\ \thetable{} --- продолжение}} \\
\hline
\Emph{Прототип функции} &\Emph{Описание функции} &\Emph{Пример использования} &\Emph{Результат}\\
\hline \hline
\endhead
\lstinline!double atof(const char*s)!& Преобразует строку в вещественное число, в случае неудачного преобразования возвращается число 0.0 &
\lstinline!char a[10];!\linebreak
\lstinline!cout<<"a=";!\linebreak
\lstinline!cin>>a;!\linebreak
\lstinline!cout<<"a="<<atof(a)<<endl;!&
\begin{verbatim}
a=23.57
a=23.57
\end{verbatim}
\\\hline
\lstinline!int atoi(const char*s)! &Преобразует строку в целое число, в случае неудачного преобразования возвращается число 0 &
\lstinline!char a[10];!\linebreak
\lstinline!cout<<"a=";!\linebreak
\lstinline!cin>>a;!\linebreak
\lstinline!cout<<"a="<<atoi(a)<<endl;!&
\begin{verbatim}
a=23
a=23
\end{verbatim}
\\\hline
\lstinline!long atol(const char*s)!&Преобразует строку в длинное целое число, в случае неудачного преобразования возвращается число 0 &
\lstinline!char a[10];!\linebreak
\lstinline!cout<<"a=";!\linebreak
\lstinline!cin>>a;!\linebreak
\lstinline!cout<<"a="<<atol(a)<<endl;!&
\begin{verbatim}
a=23
a=23
\end{verbatim}
\\\hline
\end{longtable}
}

Для преобразования числа в строку можно воспользоваться функцией \Sys{sprintf} из библиотеки
\Sys{stdio.h}.

\Sys{sprintf(s, s1, s2);}

Она аналогична описанной ранее функции \Sys{printf}, отличие состоит в том, что осуществляется вывод не на
экран, а в выходную строку~\Sys{s}.

Например, в результате работы следующих команд

\Sys{char str[80];}

\Sys{sprintf (str, \symbol{`\"}\%s \%d \%s\symbol{`\"}, \symbol{`\"}С новым \symbol{`\"}, 2014, \symbol{`\"}годом!!!\symbol{`\"});}

в переменную \Sys{str} будет записана строка \Sys{С новым 2014 годом!!!}.

\section[Тип данных string]{Тип данных \Sys{string}}\label{ch08:3}
Кроме работы со строками, как с массивом символов, в \Sys{C++} существует специальный тип данных \Sys{string}.
Для ввода переменных этого типа можно использовать \Sys{cin}\footnote{При работе c командой
\Sys{cin}, как отмечалось ранее, ввод осуществляется до пробела.} или специальную функцию:

\Sys{getline(cin,s);}

Здесь \Sys{s} --- имя вводимой переменной типа \Sys{string}.

При описании переменной типа \Sys{string} можно сразу присвоить ей значение:

\Sys{string var(s);}

Здесь \Sys{var} --- имя переменной типа \Sys{string}, \Sys{s} --- строковая
константа. В результате этого оператора создается переменная \Sys{var} типа \Sys{string} и в
нее записывается значение из строковой константы \Sys{s}. Например,

\Sys{string v(\symbol{`\"}Hello\symbol{`\"});}

Создается строка \Sys{v}, в которую записывается значение \Sys{Hello}.

Доступ к $i$-му элементу строки осуществляется стандартным образом:

\Sys{имя\_строки[номер\_элемента];}

Над строками типа \Sys{string} определены следующие \index{Строки!операции}\emph{операции}:

\begin{itemize}
\item \emph{присваивания}, например \Sys{s1=s2};
\item \emph{объединение строк} (\Sys{s1+=s2} или \Sys{s1=s1+s2}) --- добавляет к
строке \Sys{s1} строку \Sys{s2}, результат хранится в строке \Sys{s1}, например:
\begin{lstlisting}
#include <iostream>
#include <string>
using namespace std;
int main()
{
  string a,b;
  cout<<"a="; getline(cin,a);
  cout<<"b="; getline(cin,b);
  a+=b;
  cout<<"a="<<a<<endl;
  return 0;
}
\end{lstlisting}

\item \emph{сравнение строк} на основе лексикографического порядка:
\Sys{s1==s2}, \Sys{s1!=s2},
\Sys{s1<s2}, \Sys{s1<=s2}, \Sys{s1>s2},
\Sys{s1>=s2} --- результатом операций сравнения будет логическое значение;
\end{itemize}
При \emph{обработке строк }типа \Sys{string} можно использовать следующие
\index{Строки!функции}функции\footnote{В описанных ниже функциях строки \Sys{s} и \Sys{s1} должны быть типа \Sys{string}.}:
\begin{description}
\item[\Sys{s.length()}] --- возвращает длину строки \Sys{s};
\item[\Sys{s.substr(pos, length)}] --- возвращает подстроку из строки \Sys{s}, начиная с номера
 \Sys{pos} длиной \Sys{length} символов;
\item[\Sys{s.empty()}] --- возвращает значение \Sys{true}, если строка \Sys{s} пуста,
\Sys{false} --- в противном случае;
\item[\Sys{s.insert(pos, s1)}] --- вставляет строку \Sys{s1} в строку \Sys{s}, начиная с
позиции \Sys{pos};
\item[\Sys{s.remove(pos, length)}] --- удаляет из строки s подстроку \Sys{length} длиной
\Sys{pos} символов;
\item[\Sys{s.find(s1, pos)}] --- возвращает номер первого вхождения строки \Sys{s1} в строку
\Sys{s}, поиск начинается с номера \Sys{pos}, параметр \Sys{pos} может
отсутствовать, в этом случае поиск идет с начала строки;
\item[\Sys{s.findfirst(s1, pos)}] --- возвращает номер первого вхождения любого символа из строки
\Sys{s1} в строку \Sys{s}, поиск начинается с номера \Sys{pos}, параметр
\Sys{pos} может отсутствовать, в этом случае поиск идет с начала строки.
\end{description}

\prg{Некоторый текст хранится в файле \Sys{text.txt}. Подсчитать
количество строк и слов в тексте.}{ch08:prg0}

Предлагаем читателю самостоятельно разобраться в приведенном программном коде. 
\begin{lstlisting}
#include <iostream>
#include <fstream>
#include <stdlib.h>
#include <iomanip>
using namespace std;
int main()
{
  ifstream f;
  int p,j,i,kol,m,n=0;
  string S[10];
  f.open("text.txt");
  if (f)
  {
    while (!f.eof())
    {
      getline(f,S[n]);
      cout<<S[n]<<"\n";
      n++;
    }
    f.close();
    cout<<endl;
    cout<<"`\Sys{Количество строк в тексте - }`"<<n<<endl;
    for (kol=0,i=0;i<n;i++)
    {
      m=S[i].length();
      S[i]+=" ";
      for (p=0;p<m;)
      {
        j=S[i].find(" ",p);
        if (j!=0) {kol++; p=j+1;}
        else break;
      }
    }
    cout<<"`\Sys{Количество слов в тексте - }`"<<kol<<endl;
  }
  else cout<<"File not found"<<endl;
  return 0;
}
\end{lstlisting}
Результаты работы программы:
\begin{verbatim}
Если видим, что с картины
Смотрит кто-нибудь на нас,
Или принц в плаще старинном,
Или в робе верхолаз,
Лётчик или балерина,
Или Колька, твой сосед,
Обязательно картина
Называется портрет.

Количество строк в тексте - 8
Количество слов в тексте - 29
\end{verbatim}

\section[Задачи для самостоятельного решения]{Задачи для самостоятельного решения}
Разработать программу на языке \Sys{C++} для следующих заданий:

\begin{enumerate}
\item Подсчитать количество слов в каждой строке текста.
\item Подсчитать количество символов в тексте.
\item Подсчитать количество точек в тексте.
\item Подсчитать количество пробелов в тексте.
\item Удалить из теста все пробелы.
\item Удалить из теста все точки.
\item Вставить вместо каждого пробела восклицательный знак.
\item Вставить перед каждым восклицательным знаком вопросительный.
\item Определить содержит ли текст хотя бы один восклицательный знак и в какой строке.
\item Подсчитать количество слов в четных строках текста.
\item Найти номер самой длинной строки текста.
\item Променять местами первую и последнюю строки текста.
\item Определить, есть ли в тексте пустые строки.
\item Определить содержит ли текст хотя бы пару соседних одинаковых строк.
\item Найти самую короткую строку текста и заменить ее фразой «С новым годом!».
\item Найти самую длинную строку текста и заменить ее пустой строкой.
\item Определить количество слов в нечетных строках текста.
\item Определить количество пробелов в четных строках текста.
\item Определить количество предложений в тексте, учитывая, что предложение заканчивается точкой, вопросительным или
восклицательным знаком.
\item Поменять местами самую длинную и самую короткую строки текста.
\item Вывести на печать первое предложение текста, учитывая что оно заканчивается точкой.
\item Определить количество пробелов в нечетных строках текста.
\item Удалить из теста все восклицательные и вопросительные знаки.
\item Определить содержит ли текст хотя бы один вопросительный знак и в какой строке.
\item Добавить в начало каждой строки текста ее номер и пробел.
\end{enumerate}
